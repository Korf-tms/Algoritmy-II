\documentclass{article}
\usepackage[czech]{babel}
\usepackage{amsmath}
\usepackage{amssymb}
\usepackage[a4paper, margin=1in]{geometry}

\title{Domácí úkol: Žárliví manželé}
\author{}
\date{\today}

\begin{document}
	
	\maketitle
	
	\section*{Zadání}
	Úkolem je prozkoumat další z lehce bizarních úloh na překračování řeky.
	Na jednom břehu řeky se nachází $n$ manželských párů, které je třeba převézt na druhý břeh. Převoz je možný pomocí loďky, která má následující pravidla:
	
	\begin{enumerate}
		\item Loďka může převézt buď jednoho, nebo dva lidi najednou.
		\item Loďka nemůže plout sama, musí v ní vždy být alespoň jeden člověk.
		\item Manželé jsou velmi žárliví. Žádný manžel nesnese, aby jeho žena byla ve společnosti jiného muže na stejném břehu, pokud na tomto břehu není přítomna jeho manželka.
		\item Stejná pravidla platí i v loďce. Manželka a manžel se tedy mohou přepravovat společně na loďce jen pokud patří do stejného páru.
		\item Na počátku jsou všechny osoby na jednom břehu řeky, v cílovém stavu jsou všichni na druhém břehu.
		\item Manželé se mohou přemisťovat libovolně, pokud jsou dodržena pravidla o žárlivosti.	
	\end{enumerate}
	
	Vaším úkolem je \textbf{naprogramovat} řešení, které určí minimální počet převozů, které jsou potřeba k tomu, aby všech $n$ párů bylo úspěšně přepraveno na druhý břeh řeky, přičemž je nutné splnit všechna uvedená pravidla.
	Můžete, stejně jako na cvičení, použít prohledávání do šířky, nebo případně jiný způsob, který hledá nejkratší cestu v odpovídajícím stavovém grafu.
	
	
	
	\section*{Povolené a nepovolené konfigurace}
	
	Následující konfigurace znázorňují, jak mohou být jednotlivé osoby přítomny na břehu. Konfigurace, které jsou v souladu s pravidly žárlivosti, jsou označeny jako povolené, zatímco ty, které pravidla porušují, jsou označeny jako nepovolené.
	
	\begin{itemize}
		\item \texttt{husband1, husband0} – povolená konfigurace (oba manželé jsou sami bez žen)
		\item \texttt{wife0, wife1} – povolená konfigurace (obě manželky jsou samy bez mužů)
		\item \texttt{wife0, husband1} – nepovolená konfigurace (\texttt{husband1} je na břehu s cizí ženou a jeho manželka není přítomna)
		\item \texttt{wife0, wife1, husband0} – povolená konfigurace (\texttt{husband0} je s manželkou)
		\item \texttt{wife0, husband1, husband0} – povolená konfigurace (\texttt{wife0} je s manželem)
	\end{itemize}
	
	\section*{Řešení pro nízká $n$}
	
	Pro $n=2$ manželské páry lze dosáhnout převozu na druhý břeh v 5 převozech, pro 3 je třeba převozů 9, pro 4 to je 13.
	Vaším úkolem je generalizovat řešení pro obecné $n$ a to \textbf{pomocí programu}, stejně jako na cvičení.
	
	
\end{document}
